\section{Endofoncteur dynamique : version compacte}

\subsection{Cadre}

On fixe :
\begin{itemize}
  \item un type de formules $\PropT$ ;
  \item une prouvabilité relative $\Provable : \mathcal P(\PropT)\times \PropT \to \Prop$ ;
  \item des codes $\Code$, des traces $\Trace$, une exécution $\Machine:\Code\to\Trace$ ;
  \item un kit $K$ et une certification $\Rev0_K(K,\cdot):\Trace\to\Prop$ ;
  \item un encodage $\encodehalt:\Code\to\PropT$ ;
  \item une clôture $Cn:\mathcal P(\PropT)\to \mathcal P(\PropT)$.
\end{itemize}

\paragraph{Hypothèses.}
\begin{description}
  \item[(PMon)] \textbf{Monotonie de la prouvabilité :}
  $\Gamma\subseteq\Delta \Rightarrow \big(\Provable(\Gamma,p)\Rightarrow \Provable(\Delta,p)\big)$.
  \item[(CnMon)] \textbf{Monotonie de $Cn$ :}
  $\Gamma\subseteq\Delta \Rightarrow Cn(\Gamma)\subseteq Cn(\Delta)$.
  \item[(CnIdem)] \textbf{Idempotence :} $Cn(Cn(\Gamma))=Cn(\Gamma)$.
  \item[(CnExt)] \textbf{Extensivité :} $\Gamma\subseteq Cn(\Gamma)$.
  \item[(PCn)] \textbf{Production de $\ProvClosed$ :} $\forall \Gamma,\ \ProvClosed(Cn(\Gamma))$.
\end{description}
On travaille en logique classique quand un raisonnement par cas sur $\Provable(\Delta,p)$ est utilisé.

\paragraph{Admissibilité.}
On définit :
\[
\ProvClosed(\Gamma)\;:\!\iff\;\forall p,\ \Provable(\Gamma,p)\to p\in\Gamma.
\]
On définit aussi l’absorbabilité :
\[
\Absorbable(\Gamma)\;:\!\iff\;\forall p,\ p\in\Gamma\to \Provable(\Gamma,p).
\]

\subsection{Frontière dynamique et pas}

\paragraph{Frontière.}
Pour $\Gamma\subseteq\PropT$, on pose :
\[
S_1(\Gamma)=
\left\{
p\ \middle|\ 
\exists e\in\Code,\ 
p=\encodehalt(e)\ \wedge\ \Rev0_K(K,\Machine(e))\ \wedge\ \neg \Provable(\Gamma,\encodehalt(e))
\right\}.
\]

\paragraph{Pas.}
On pose :
\[
F_0(\Gamma):=\Gamma\cup S_1(\Gamma),
\qquad
F(\Gamma):=Cn(F_0(\Gamma))=Cn(\Gamma\cup S_1(\Gamma)).
\]

\medskip
\noindent
\textbf{Point central.} $S_1(\Gamma)$ dépend de $\Gamma$ via $\neg\Provable(\Gamma,\cdot)$ : c’est une \emph{frontière recalculée} à chaque pas. Le caractère non trivial est que $S_1$ est anti-monotone, mais $F$ peut tout de même devenir fonctoriel sur une classe d’états admissibles.

\subsection{États admissibles et catégorie mince}

On définit la catégorie (préordre) $\ThState$ :
\begin{itemize}
  \item objets : $A=(\Gamma,\ Cn(\Gamma)=\Gamma,\ \ProvClosed(\Gamma))$ ;
  \item morphismes $A\to B$ : inclusions $A.\Gamma\subseteq B.\Gamma$.
\end{itemize}
La catégorie est mince : entre deux objets il y a au plus une flèche.

\subsection{Lemme-clé : monotonie relative du pas minimal}

\begin{lemme}[Monotonie relative de $F_0$]
Si $\Gamma\subseteq\Delta$ et $\ProvClosed(\Delta)$, alors
\[
F_0(\Gamma)\subseteq F_0(\Delta).
\]
\end{lemme}

\begin{proof}[Idée]
Soit $p\in F_0(\Gamma)$.
\begin{itemize}
  \item Si $p\in\Gamma$, alors $p\in\Delta$, donc $p\in F_0(\Delta)$.
  \item Si $p\in S_1(\Gamma)$, on a $p=\encodehalt(e)$, $\Rev0_K(K,\Machine(e))$ et $\neg\Provable(\Gamma,p)$.
  Par cas sur $\Provable(\Delta,p)$ :
  \begin{itemize}
    \item Si $\Provable(\Delta,p)$, alors $\ProvClosed(\Delta)$ implique $p\in\Delta\subseteq F_0(\Delta)$ ;
    \item Sinon $\neg\Provable(\Delta,p)$, donc $p\in S_1(\Delta)\subseteq F_0(\Delta)$.
  \end{itemize}
\end{itemize}
\end{proof}

\subsection{Endofoncteur dynamique}

\begin{theoreme}[Endofoncteur]
Sous (CnIdem), (CnMon), (PCn), l’application
\[
\Step(A).\Gamma := F(A.\Gamma)=Cn(A.\Gamma\cup S_1(A.\Gamma))
\]
définit un endofoncteur
\[
\TheoryStepFunctor:\ThState\to\ThState.
\]
\end{theoreme}

\begin{proof}[Esquisse]
\emph{Objets :} (CnIdem) donne $Cn(\Step(A).\Gamma)=\Step(A).\Gamma$ et (PCn) donne $\ProvClosed(\Step(A).\Gamma)$.

\emph{Morphismes :} si $A.\Gamma\subseteq B.\Gamma$, alors $B.\ProvClosed$ donne $F_0(A.\Gamma)\subseteq F_0(B.\Gamma)$ par le lemme précédent, puis (CnMon) donne $F(A.\Gamma)\subseteq F(B.\Gamma)$.

\emph{Lois fonctorielles :} trivial dans une catégorie mince.
\end{proof}

\subsection{Dynamique itérée}

\begin{definition}[Chaîne]
Pour $A_0\in\ThState$, on définit :
\[
A_{0}:=A_0,\qquad A_{n+1}:=\TheoryStepFunctor(A_n).
\]
Donc
\[
A_{n+1}.\Gamma = Cn\big(A_n.\Gamma\cup S_1(A_n.\Gamma)\big).
\]
\end{definition}

\subsection{Phénomènes : croissance stricte, $\omega$-collapse, trilemme}

\paragraph{Témoin de frontière.}
On dira que $\Gamma$ admet un témoin si :
\[
\FrontierWitness(\Gamma)\;:\!\iff\;\exists e,\ \Rev0_K(K,\Machine(e))\ \wedge\ \neg\Provable(\Gamma,\encodehalt(e)).
\]

\begin{theoreme}[Croissance stricte sous absorbabilité + témoin]
Sous (CnExt), si $\Absorbable(\Gamma)$ et $\FrontierWitness(\Gamma)$, alors
\[
\Gamma \subset F(\Gamma).
\]
\end{theoreme}

\begin{proof}[Idée]
Le témoin donne $p=\encodehalt(e)\in S_1(\Gamma)\subseteq \Gamma\cup S_1(\Gamma)$, donc $p\in F(\Gamma)$ par (CnExt).
Si $\Gamma=F(\Gamma)$, alors $p\in\Gamma$, donc $\Provable(\Gamma,p)$ par $\Absorbable(\Gamma)$, contradiction avec la définition de $S_1(\Gamma)$.
\end{proof}

\paragraph{Limite $\omega$.}
On pose le porteur limite :
\[
\omega\Gamma := \bigcup_{n\in\mathbb N} A_n.\Gamma
\quad\text{(i.e. } p\in\omega\Gamma \Leftrightarrow \exists n,\ p\in A_n.\Gamma\text{)}.
\]

\begin{theoreme}[$\omega$-collapse]
Sous (PMon), (CnExt), (CnIdem), (PCn), si $\Absorbable(A_1.\Gamma)$ alors
\[
S_1(\omega\Gamma)=\varnothing.
\]
\end{theoreme}

\begin{proof}[Idée]
Supposons $p=\encodehalt(e)\in S_1(\omega\Gamma)$, donc $\neg\Provable(\omega\Gamma,p)$.
Par contraposée de (PMon) et $A_0.\Gamma\subseteq \omega\Gamma$, on a $\neg\Provable(A_0.\Gamma,p)$, donc $p\in S_1(A_0.\Gamma)$.
Ainsi $p\in A_0.\Gamma\cup S_1(A_0.\Gamma)$ et donc $p\in A_1.\Gamma$ par (CnExt).
Puis $\Absorbable(A_1.\Gamma)$ donne $\Provable(A_1.\Gamma,p)$, et (PMon) via $A_1.\Gamma\subseteq \omega\Gamma$ donne $\Provable(\omega\Gamma,p)$, contradiction.
\end{proof}

\paragraph{Route II et admissibilité à $\omega$.}
On définit :
\[
\OmegaAdmissible(\omega\Gamma)\;:\!\iff\;Cn(\omega\Gamma)=\omega\Gamma\ \wedge\ \ProvClosed(\omega\Gamma),
\qquad
\RouteIIAt(\omega\Gamma)\;:\!\iff\; S_1(\omega\Gamma)\neq\varnothing.
\]

\begin{theoreme}[Trilemme à $\omega$]
Sous (PMon), (CnExt), (CnIdem), (PCn), on a :
\[
\neg \Absorbable(A_1.\Gamma)
\ \ \lor\ \
\neg \OmegaAdmissible(\omega\Gamma)
\ \ \lor\ \
\neg \RouteIIAt(\omega\Gamma).
\]
\end{theoreme}

\begin{proof}[Idée]
Si $\Absorbable(A_1.\Gamma)$ tenait, alors le théorème de $\omega$-collapse donne $S_1(\omega\Gamma)=\varnothing$, donc $\neg\RouteIIAt(\omega\Gamma)$.
\end{proof}

\subsection{Conclusion (message structurel)}
Le pas
\[
\Gamma \longmapsto Cn\big(\Gamma\cup S_1(\Gamma)\big)
\]
couple une clôture statique $Cn$ à une frontière dynamique $S_1$ définie via $\neg\Provable(\Gamma,\cdot)$.
La restriction aux états admissibles (où la prouvabilité s’absorbe en appartenance) rend ce pas \emph{fonctoriel}.
Cette fonctorialité rend ensuite accessibles des phénomènes intrinsèquement dynamiques (croissance stricte, collapse à $\omega$, trilemme) qui n’apparaissent pas dans une lecture purement statique des théories.
