\section{Endofoncteur dynamique sur les états admissibles}

\subsection{Données et notations}
On fixe :
\begin{itemize}
  \item un type de formules $\PropT$ ;
  \item une prouvabilité relative $\Provable : \mathcal P(\PropT)\times \PropT \to \Prop$ ;
  \item des codes $\Code$, des traces $\Trace$, et une exécution $\Machine:\Code\to\Trace$ ;
  \item un kit $K$ et un prédicat de certification $\Rev0_K(K,\cdot):\Trace\to\Prop$ ;
  \item un encodage $\encodehalt:\Code\to\PropT$ (``ce code s’arrête'') ;
  \item une clôture $Cn:\mathcal P(\PropT)\to \mathcal P(\PropT)$.
\end{itemize}

\paragraph{Axiomes (structure).}
On supposera typiquement :
\begin{description}
  \item[(PMon)] \textbf{Monotonie de la prouvabilité :}
  $\Gamma\subseteq\Delta \Rightarrow \big(\Provable(\Gamma,p)\Rightarrow \Provable(\Delta,p)\big)$.
  \item[(CnExt)] \textbf{Extensivité :} $\Gamma\subseteq Cn(\Gamma)$.
  \item[(CnMon)] \textbf{Monotonie :} $\Gamma\subseteq\Delta \Rightarrow Cn(\Gamma)\subseteq Cn(\Delta)$.
  \item[(CnIdem)] \textbf{Idempotence :} $Cn(Cn(\Gamma))=Cn(\Gamma)$.
  \item[(PCn)] \textbf{Production de $\ProvClosed$ :} $\forall \Gamma,\ \ProvClosed(Cn(\Gamma))$.
\end{description}
On travaille en logique classique lorsque l’on utilise un raisonnement par cas sur $\Provable(\Delta,p)$.

\paragraph{Admissibilité.}
On définit :
\[
\ProvClosed(\Gamma) \;:\!\iff\; \forall p,\ \Provable(\Gamma,p)\to p\in\Gamma.
\]
On définit aussi l’``absorbabilité'' :
\[
\Absorbable(\Gamma)\;:\!\iff\;\forall p,\ p\in\Gamma\to \Provable(\Gamma,p).
\]
(En pratique, $\Absorbable$ est une moitié de ``post-splitter'' : appartenance $\Leftrightarrow$ prouvabilité.)

\subsection{Frontière dynamique et pas sur corpus}
Pour $\Gamma\subseteq \PropT$, on définit la \emph{frontière relative} :
\[
S_1(\Gamma)
=
\left\{
p\ \middle|\ 
\exists e\in\Code,\ 
p=\encodehalt(e)\ \wedge\ \Rev0_K(K,\Machine(e))\ \wedge\ \neg \Provable(\Gamma,\encodehalt(e))
\right\}.
\]
On pose ensuite :
\[
F_0(\Gamma) := \Gamma \cup S_1(\Gamma),
\qquad
F(\Gamma) := Cn(F_0(\Gamma))=Cn(\Gamma\cup S_1(\Gamma)).
\]
Le caractère \emph{dynamique} vient du fait que $S_1(\Gamma)$ dépend de $\Gamma$ via $\neg\Provable(\Gamma,\cdot)$ et doit être recalculée à chaque pas.

\subsection{Catégorie mince des états admissibles}
On définit $\ThState$ comme la catégorie (préordre) dont :
\begin{itemize}
  \item les objets sont les états admissibles $A=(\Gamma,\ Cn(\Gamma)=\Gamma,\ \ProvClosed(\Gamma))$ ;
  \item les morphismes $A\to B$ sont les inclusions $A.\Gamma\subseteq B.\Gamma$.
\end{itemize}
La catégorie est mince : entre deux objets, il y a au plus une flèche.

\subsection{Résultats principaux}

\begin{proposition}[Anti-monotonie de la frontière \,(\texttt{S1Rel\_anti\_monotone})]
Sous (PMon), si $\Gamma\subseteq\Delta$ alors
\[
S_1(\Delta)\subseteq S_1(\Gamma).
\]
\end{proposition}

\begin{proof}[Idée]
Si $p\in S_1(\Delta)$, alors $p=\encodehalt(e)$, $\Rev0_K(K,\Machine(e))$ et $\neg\Provable(\Delta,p)$.
Si $\Provable(\Gamma,p)$, alors par (PMon) on aurait $\Provable(\Delta,p)$, contradiction.
Donc $\neg\Provable(\Gamma,p)$ et $p\in S_1(\Gamma)$.
\end{proof}

\begin{lemme}[Monotonie relative de $F_0$ \,(\texttt{F0\_monotone\_of\_provClosed})]
Si $\Gamma\subseteq\Delta$ et $\ProvClosed(\Delta)$, alors
\[
F_0(\Gamma)\subseteq F_0(\Delta).
\]
\end{lemme}

\begin{proof}[Schéma]
Soit $p\in F_0(\Gamma)$.
\begin{itemize}
  \item Si $p\in\Gamma$, alors $p\in\Delta$, donc $p\in F_0(\Delta)$.
  \item Si $p\in S_1(\Gamma)$, on a $p=\encodehalt(e)$, $\Rev0_K(K,\Machine(e))$ et $\neg\Provable(\Gamma,p)$.
  Par cas sur $\Provable(\Delta,p)$ :
  \begin{itemize}
    \item Si $\Provable(\Delta,p)$, alors $\ProvClosed(\Delta)$ implique $p\in\Delta$, donc $p\in F_0(\Delta)$.
    \item Sinon $\neg\Provable(\Delta,p)$, donc $p\in S_1(\Delta)\subseteq F_0(\Delta)$.
  \end{itemize}
\end{itemize}
\end{proof}

\begin{theoreme}[Endofoncteur dynamique \,(\texttt{TheoryStepFunctor})]
Sous (CnIdem), (CnMon), (PCn), l’application
\[
\Step(A).\Gamma := F(A.\Gamma)=Cn(A.\Gamma\cup S_1(A.\Gamma))
\]
définit un endofoncteur
\[
\TheoryStepFunctor:\ThState\longrightarrow \ThState.
\]
\end{theoreme}

\begin{proof}[Idée]
\emph{(Bien-définition sur objets).} (CnIdem) donne $Cn(\Step(A).\Gamma)=\Step(A).\Gamma$, et (PCn) donne $\ProvClosed(\Step(A).\Gamma)$.

\emph{(Action sur morphismes).} Si $A.\Gamma\subseteq B.\Gamma$, alors $B.\ProvClosed$ et le lemme précédent donnent
$F_0(A.\Gamma)\subseteq F_0(B.\Gamma)$, puis (CnMon) donne
$Cn(F_0(A.\Gamma))\subseteq Cn(F_0(B.\Gamma))$.

\emph{(Lois fonctorielles).} La catégorie étant mince, \texttt{map\_id} et \texttt{map\_comp} sont immédiats.
\end{proof}

\begin{definition}[Chaîne itérée \,(\texttt{chainState})]
Pour $A_0\in\ThState$, on définit
\[
A_{0}:=A_0,\qquad
A_{n+1}:=\TheoryStepFunctor(A_n).
\]
Ainsi,
\[
A_{n+1}.\Gamma = Cn\big(A_n.\Gamma \cup S_1(A_n.\Gamma)\big).
\]
\end{definition}

\subsection{Invariants et phénomènes de limite}

\begin{definition}[Défaut relatif]
Pour $\Gamma\subseteq\PropT$ et $\Delta\subseteq\PropT$, on pose
\[
\MissingFrom(\Gamma,\Delta):=\{p\mid p\in\Delta\ \wedge\ \neg\Provable(\Gamma,p)\}.
\]
\end{definition}

\begin{proposition}[``Conservation'' : $\MissingFrom(\Gamma,F_0(\Gamma))=S_1(\Gamma)$ \,(\texttt{missing\_F0\_eq\_S1\_of\_absorbable})]
Si $\Absorbable(\Gamma)$, alors
\[
\MissingFrom(\Gamma,F_0(\Gamma))=S_1(\Gamma).
\]
\end{proposition}

\begin{proof}[Idée]
Si $p\in \MissingFrom(\Gamma,F_0(\Gamma))$, alors $p\in F_0(\Gamma)=\Gamma\cup S_1(\Gamma)$ et $\neg\Provable(\Gamma,p)$.
Le cas $p\in\Gamma$ contredit $\Absorbable(\Gamma)$, donc $p\in S_1(\Gamma)$.
La réciproque est immédiate : $S_1(\Gamma)\subseteq F_0(\Gamma)$ et, par définition, $p\in S_1(\Gamma)\Rightarrow \neg\Provable(\Gamma,p)$.
\end{proof}

\begin{theoreme}[Croissance stricte sous témoin \,(\texttt{strict\_F\_of\_absorbable} / \texttt{strict\_step\_state})]
Sous (CnExt), si $\Absorbable(\Gamma)$ et il existe un témoin de frontière
\[
\exists e,\ \Rev0_K(K,\Machine(e))\ \wedge\ \neg\Provable(\Gamma,\encodehalt(e)),
\]
alors
\[
\Gamma \subset F(\Gamma).
\]
\end{theoreme}

\begin{proof}[Idée]
Le témoin donne $\encodehalt(e)\in S_1(\Gamma)$, donc $\encodehalt(e)\in \Gamma\cup S_1(\Gamma)$ puis, par (CnExt), $\encodehalt(e)\in Cn(\Gamma\cup S_1(\Gamma))=F(\Gamma)$.
Si $\Gamma=F(\Gamma)$, alors $\encodehalt(e)\in\Gamma$, donc par $\Absorbable(\Gamma)$ on aurait $\Provable(\Gamma,\encodehalt(e))$, contradiction.
\end{proof}

\begin{definition}[Limite $\omega$ (porteur)]
On définit le porteur limite :
\[
\omega\Gamma := \bigcup_{n\in\mathbb N} A_n.\Gamma
\quad
\text{(i.e. } p\in\omega\Gamma \Leftrightarrow \exists n,\ p\in A_n.\Gamma\text{)}.
\]
\end{definition}

\begin{theoreme}[$\omega$-collapse \,(\texttt{S1Rel\_omegaΓ\_eq\_empty\_of\_absorbable\_succ})]
Sous (PMon), (CnExt), (CnIdem), (PCn), si $\Absorbable(A_1.\Gamma)$ alors
\[
S_1(\omega\Gamma)=\varnothing.
\]
\end{theoreme}

\begin{proof}[Idée]
Supposons $p=\encodehalt(e)\in S_1(\omega\Gamma)$. Alors $\neg\Provable(\omega\Gamma,p)$.
Par contraposée de (PMon) appliquée à $A_0.\Gamma\subseteq \omega\Gamma$, on obtient $\neg\Provable(A_0.\Gamma,p)$, donc $p\in S_1(A_0.\Gamma)$.
Ainsi $p\in A_0.\Gamma\cup S_1(A_0.\Gamma)$, puis par (CnExt) on a $p\in A_1.\Gamma$.
Par $\Absorbable(A_1.\Gamma)$ on obtient $\Provable(A_1.\Gamma,p)$, puis (PMon) via $A_1.\Gamma\subseteq \omega\Gamma$ donne $\Provable(\omega\Gamma,p)$, contradiction.
\end{proof}

\begin{definition}[Admissibilité à $\omega$ et Route II]
On pose :
\[
\OmegaAdmissible(\omega\Gamma)\;:\!\iff\; Cn(\omega\Gamma)=\omega\Gamma \ \wedge\ \ProvClosed(\omega\Gamma),
\]
et
\[
\RouteIIAt(\omega\Gamma)\;:\!\iff\; S_1(\omega\Gamma)\neq\varnothing.
\]
\end{definition}

\begin{theoreme}[Trilemme à $\omega$ \,(\texttt{omega\_trilemma\_disjunction})]
Sous (PMon), (CnExt), (CnIdem), (PCn), pour toute orbite issue de $A_0$ on a :
\[
\neg \Absorbable(A_1.\Gamma)
\ \ \lor\ \
\neg \OmegaAdmissible(\omega\Gamma)
\ \ \lor\ \
\neg \RouteIIAt(\omega\Gamma).
\]
\end{theoreme}

\begin{proof}[Idée]
Si $\Absorbable(A_1.\Gamma)$ et $\OmegaAdmissible(\omega\Gamma)$ tenaient, alors le théorème de $\omega$-collapse impose $S_1(\omega\Gamma)=\varnothing$, donc $\neg\RouteIIAt(\omega\Gamma)$.
\end{proof}

\subsection{Lecture (``punchline'')}
La nouveauté n’est pas la clôture $Cn$ (statique), mais le couplage dynamique
\[
\Gamma \mapsto Cn\big(\Gamma\cup S_1(\Gamma)\big),
\]
où $S_1$ dépend de $\neg\Provable(\Gamma,\cdot)$.
La restriction aux états admissibles rend ce pas \emph{fonctoriel}, et permet ensuite d’obtenir des phénomènes intrinsèquement dynamiques (croissance stricte, collapse à $\omega$, trilemmes structuraux) invisibles dans une présentation purement statique.
